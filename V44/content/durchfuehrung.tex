\section{Durchführung}
\label{sec:Durchführung}
Der Versuch wurde an einem D8-Labordiffraktometer durchgeführt.
Eine Röntgenröhre, die mit \qty{40}{\kilo\volt} und \qty{35}{\milli\ampere} in Betrieb ist,
bestrahlt eine Probe. Ein Detektor erfasst dabei die reflektierte Strahlung.
Die Probe, ein mit Polymerfilm beschichteter Siliziumwafer, muss durch eine Vielzahl von Scans zunächst justiert werden

\subsection{Justierung}
\subsubsection*{Detektorscan}
Unanbhängig von der Probe werden Röhre und Detektor auf eine Ebene gebracht. Nun fährt der Detektor einen kleinen Winkelbereich um den Strahl der Röhre ab.
Das Maximum der aufgenommenen Intensität, die einer Gaußglocke ähnelt, ist fortan der neue Nullpunkt der Detektorausrichtung.

\subsubsection*{Z-Scan}
Der Z-Scan dient zur Höhenausrichtung der Probe.
Dabei wird die Probe leicht in der Höhe verschoben.
Richtig justiert ist sie,
wenn sie parallel zum Strahl steht (siehe Detektorscan) und die halbe Intensität des Primärstrahls abschattet.
Die Intensität des Strahls nimmt desto mehr ab, je weiter sich die Probe im Strahl befindet.
Die Probe wird soweit in den Strahl gefahren, bis eine halbe Abschattung zu erkennen ist. 

\subsubsection*{Rockingscan}
Dennoch besteht die Möglichkeit, dass die Probe nicht parallel zur Ausbreitungsrichtung des Röntgenstrahls steht.
Demnach wird die Probe ungleichmäßig getroffen. Um auch diese Hürde zu bewältigen drehen sich beim Rockingscan
Röhre und Detektor mit konstanter Winkelsumme um die Probe um diese im Drehpunkt des Diffraktometers zu zentrieren.

\subsubsection*{Feinjustierung}
Nach den ersten drei Scans folgt ein weiter Z-Scan um nach der Drehung wieder die Hälfte abzudecken.
Anschließend wird noch einmal der Rockingscan mit einem abschließendem Z-Scan wiederholt um endgültig mit der Messung anzufangen.

\subsection{Messung}
Gemessen werden ein Reflektivitätsscan und ein diffuser Scan.
Hierzu wird das eingerichtete Programm, XRD Commander, gemäß der Anleitung\cite{sample} genutzt.

Beim Reflektivitätsscan sind der Röhren- und Detektorwinkel gleich groß. Es wird in
einem Bereich von 0° bis 2,5° mit einer Schrittweite von 0,005° und einer Messzeit von
\qty{5}{\second} gemessen.
Bei dem diffusen Scan ist der Detektorwinkel um 0,1° gegenüber dem Röhrenwinkel
verschoben. Jedoch bleibt der Vorgang der gleiche. Durch diesen Scan, kann später die wahre Reflektivität berechnet werden.