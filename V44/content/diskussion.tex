\section{Diskussion}
\label{sec:Diskussion}

Der bei der Justierung des Strahls gemessene Geometriewinkel $\alpha = 0.4^°$ weicht leicht vom Literaturwert $\alpha_{t} = 0,687^°$ ab.
Somit war die Justierung erfolgreich, wenn auch nicht optimal.
Auch die über die Kiessig-Oszillation berechnete Schichtdicke weicht nur geringfügig zwischen dem gemessenen 
und den durch Parrat berechneten Werten ab.
\begin{align*}
    d_{Kiessig} &= 882,4 \text{ \AA} \\
    d_{Parrat} &= 860 \text{ \AA}
\end{align*}
Dies folgt wohl der Verbesserungsmöglichkeit der Justierung.
Ebenso ist die zerkratzte Wafer-Oberfläche nicht zu vernachlässigen.
Gerade beim Einstellen der Rauigkeitsparameter des Parrat Algorithmus spielt sie eine entscheidene Rolle.
Dies lässt sich auch an den Dispersionswerten erkennen.
\begin{align*}
    \delta_{Poly} &= 0.3 \cdot 10^{-6}  \\
    \delta_{Si} &= 6.3 \cdot 10^{-6} 
\end{align*}
Die Literaturwerte \cite{wert} liegen bei:
\begin{align*}
    \delta_{Poly} &= 3.5 \cdot 10^{-6}  \\
    \delta_{Si} &= 7.6 \cdot 10^{-6} 
\end{align*}
Während die Abweichung von Silizium bei $\qty{20}{\percent}$ liegt, beträgt die von Polystrol fas $\qty{1000}{\percent}$,
was auf einen systematischen Fehler hinweißt.
Diese hohe Unsicherheit bei der Dispersion von Polystrol erkennt man auch im Vergleich der kritischen Winkel.
\begin{align*}
    \alpha_{Poly} &= 0,068^°  \\
    \alpha_{Si} &= 0,214^°
\end{align*}
Die Literaturwerte \cite{wert} liegen bei:
\begin{align*}
    \alpha_{Poly} &= 0,153^°  \\
    \alpha_{Si} &= 0,223^°
\end{align*}
Insgesamt liegt eine gewisse Fehleranfälligkeit auf das Einstellen der Parratt-Parameter per Hand.
Viel erheblicher fällt die nicht oütimal glatte Oberfläche der Probe ins Gewicht,
welche durch ihre hohe Rauigkeit für Messfehler sorgt. 
Die Parrat-Kurve beschreibt die gemessene Reflektivität aber genau genug, 
so dass dem Ergebnis eine Aussagekraft zugeschrieben werden kann.