\section{Diskussion}
\label{sec:Diskussion}

Der bei der Justirung des Strahls gemessene Geometriewinkel $\alpha = 0.4^°$ weicht leicht vom Literaturwert $\alpha_{t} = 0,687^°$ ab.
Somit war die Justierung erfolgreich, wenn auch nicht optimal.
Die über die Kiessig-Oszillation berechnete Schichtdicke weicht allerdings zwischen dem gemessenen 
und den durch Parrat berechneten Wert stark ab.
\begin{align*}
    d_{Kiessig} &= 4.99 \cdot 10^{-8} m \\
    d_{Parrat} &= 8.6 \cdot 10^{-8} m
\end{align*}
Dies folgt wohl der Verbesserungsmöglichkeit der Justierung.
Ebenso ist die zerkratzte Wafer Oberfläche nicht zu vernachlässigen,
gerade beim einstellen der Rauigkeitsparrameter beim Parrat Algorithmus spielt dies eine Entscheidene Rolle.
Dies lässt sich auch an den Dispersionswerten erkennen.
\begin{align*}
    \delta_{Poly} &= 7 \cdot 10^{-7}  \\
    \delta_{Si} &= 7 \cdot 10^{-6} 
\end{align*}
Die Literaturwerte \cite{wert} liegen bei:
\begin{align*}
    \delta_{Poly} &= 3.5 \cdot 10^{-6}  \\
    \delta_{Si} &= 7.6 \cdot 10^{-6} 
\end{align*}
Während die Abweichung von Silizium bei $\qty{7}{\percent}$ liegt beträgt die von Polystrol fast $\qty{400}{\percent}$,
was auf einen systematischen Fehler hinweißt.
Diese hohe Unsicherheit bei der Dispersion von Polystrol erkennt man auch im Vergleich der kritischen Winkel.
\begin{align*}
    \alpha_{Poly} &= 0,068^°  \\
    \alpha_{Si} &= 0,214^°
\end{align*}
Die Literaturwerte \cite{wert} liegen bei:
\begin{align*}
    \alpha_{Poly} &= 0,153^°  \\
    \alpha_{Si} &= 0,223^°
\end{align*}

Insgesammt liegt eine gewisse Fehleranfälligkeit auf das per Handeinstellen der Parratt-Parameter.
Viel erheblicher fällt die nicht obtimal glatte Proben Oberfläche ins gewicht,
welche durch ihre hohe Rauigkeit für Messfehler sorgt. 
Die Parrat-Kurve beschreibt die gemessene Reflektivität aber genau genug, 
so dass dem Ergebniss eine Aussagekraft zugeschrieben werden kann.