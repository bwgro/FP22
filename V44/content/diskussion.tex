\section{Diskussion}
\label{sec:Diskussion}

Der bei der Justirung des Strahls gemessene Geometriewinkel $\alpha = 0.4^°$ weicht leicht vom Literatur Wert $\alpha_{t} = 0,687^°$ ab.
Somit war die Justierung Erfolgreich, wenn auch nicht optimal.
Die über die Kiesling-Osszilation berechnete Schichtdicke weicht allerdings zwischen dem Gemessenen 
und  den durch Parrat berechneten Wert stark ab.
\begin{align*}
    d_{Kiessig} &= 1,71 \cdot 10^{-8} m \\
    d_{Parrat} &= 1,7 \cdot 10^{-6} m
\end{align*}
Dies kann sich aus der nicht perfekten Justierung der Probe ergeben.
Ebendso sind die Parameter des Parrats Algorithmus durch Hand eingestelllt und somit relativ Fehler anffällig.
Dies Lässt sich auch an den Dispersionswerten erkennen.
\begin{align*}
    \delta_{Poly} &= 7 \cdot 10^{-7}  \\
    \delta_{Si} &= 7 \cdot 10^{-6} 
\end{align*}
Die Literatur Werte \cite{wert} liegen bei:
\begin{align*}
    \delta_{Poly} &= 3.5 \cdot 10^{-6}  \\
    \delta_{Si} &= 7.6 \cdot 10^{-6} 
\end{align*}
Während die Abweichung von Silizium bei $7\%$ liegt beträgt die von Polystrol fast $400\%$,
was auf einen Systematischen Fehler hinweißt.
Diese hohe Unsicherheit bei der Dispersion von Polystrol erkennt man auch im vergleich der kritischen Winkel.
\begin{align*}
    \alpha_{Poly} &= 0,068^°  \\
    \alpha_{Si} &= 0,214^°
\end{align*}
Die Literatur Werte \cite{wert} liegen bei:
\begin{align*}
    \alpha_{Poly} &= 0,153^°  \\
    \alpha_{Si} &= 0,223^°
\end{align*}

Insgesammt liegt eine große Fehleranfälligkeit auf das per Handeinstellen der Parratt Parameter.
Die Parrat Kurve beschreibt die gemessene Reflektivität aber genau genug, 
so dass dem Ergebniss eine Aussagekraft zugeschrieben werden kann.