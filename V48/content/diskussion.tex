\section{Diskussion}
\label{sec:Diskussion}
Das Ablesen am Picoamperemeter gestaltete sich äußerst schwierig. Vor allem bei der zweiten Messung ist dies festzustellen.
Der Kurvenverlauf des Depolarisationsstroms wurde nur undeutlich aufgenommen. Die äußere Struktur eines Maximas ist zwar zu erkennen,
jedoch ebenso wie die unvorhersehbaren Ausschläge des verwendeten analogen Messgerätes, die bereits durch geringe Einflüsse auf den Versuchsaufbau hervorgerufen wurden.

In \autoref{tab:W} werden die verschiedenen Auswertungsmethoden der jeweiligen Messung miteinander verglichen.
\begin{table}[h]
    \centering
    \caption{Direkter Vergleich der Aktivierungsenergien.}
    \label{tab:W}
    \begin{tabular}{cSS}
        \toprule
        {Ansatz} & {Messreihe 1} & {Messreihe 2}\\
        \cmidrule(lr){2-2}\cmidrule(lr){3-3}
        &{$W_1 \, / \, \si{\electronvolt}$} & {$W_2 \, / \, \si{\electronvolt}$} \\
        \midrule
        {Polarisation} & {$0.680(41)$} & {$0.692(31)$} \\   
        {Stromdichte}  & {$0.538(181)$} & {$0.533(93)$} \\
        \bottomrule
        \end{tabular}
\end{table}
\noindent
Die relative Abweichungen
\begin{equation}
    \Delta x = \left| \frac{x_\text{exp} - x_\text{lit}}{x_\text{lit}}\right|
\end{equation}
der Werte des Polarisationsansatzes vom Literaturwert $W_\text{lit} = \qty{0.66(1)}{\electronvolt}$\cite{Buch} liegen bei:
\begin{align}
    \Delta W_1 = \qty{3}{\percent} \\
    \Delta W_2 = \qty{5}{\percent}
\end{align}
Bei der ersten Messung liegt der Literaturwert in den Fehlergrenzen, jedoch ist er bei der zweiten Messung knapp außerhalb.
Dies kann unteranderem auch Rückschlüsse zum fehlerbehafteten Ablesen am Picoamperemeter ziehen.
Dahingegen liegen sie bei der Auswertung mittels des Stromdichtenansatzes bei:
\begin{align}
    \Delta W_1 = \qty{18.5}{\percent} \\
    \Delta W_2 = \qty{19.3}{\percent}
\end{align}
Dieser doch bemerkliche Unterschied der beiden Methoden spiegelt gegebenenfalls die Natur der numerischen Integration gegenüber der
weniger komplexen Methode der Approximation wieder. 

Im Gegensatz dazu weichen die charakteristischen Relaxationszeiten weiter von einander ab (vgl. Tab. \ref{tab:t}).
Dies ist aber auf eine zusätzliche andere Vorhergehensweise in der Ausgleichsrechnung des Stromdichtenansatzes zurückzuführen.
Wendet man anstelle der angewandten Umformung ebenfalls \autoref{eq:relax} an, so verringern auch hier sich die anliegenden Unterschiede.
\begin{table}[h]
    \centering
    \caption{Direkter Vergleich der charakteristischen Relaxationszeiten.}
    \label{tab:t}
    \begin{tabular}{cSS}
        \toprule
        {Ansatz} & {Messreihe 1} & {Messreihe 2} \\
        \cmidrule(lr){2-2}  \cmidrule(lr){3-3} 
        & {$\tau \, / \, \si{\second}$} & {$\tau \, / \, \si{\second}$} \\
        \midrule
        {Polarisation} & {$2.25(4.18)e-11$} & {$4.96(6.90)e-11$} \\   
        {Stromdichte}  & {$1.97(3.70)e-11$} & {$4.52(6.35)e-11$} \\
        \bottomrule
    \end{tabular}
\end{table}
