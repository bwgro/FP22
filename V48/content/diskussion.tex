\section{Diskussion}
\label{sec:Diskussion}
Das Ablesen am Picoamperemeter gestaltete sich äußerst schwierig. Vor allem bei der zweiten Messung ist dies festzustellen.
Der Kurvenverlauf des Depolarisationsstroms wurde nur undeutlich aufgenommen,
jedoch ist der erwartete exponentielle Verlauf im Groben zu erkennen.
Bereits die kleinsten äußerlichen Einwirkungen auf den Versuchsaufbau haben das Messergebnis unvorhersehbar verfälscht.
In \autoref{tab:W} werden die verschiedenen Auswertungsmethoden der jeweiligen Messung miteinander verglichen.
\begin{table}[h]
    \centering
    \caption{Direkter Vergleich der Aktivierungsenergien.}
    \label{tab:W}
    \begin{tabular}{cSS}
        \toprule
        {Ansatz} & {Messreihe 1} & {Messreihe 2}\\
        \cmidrule(lr){2-2}\cmidrule(lr){3-3}
        &{$W_1 \, / \, \si{\electronvolt}$} & {$W_2 \, / \, \si{\electronvolt}$} \\
        \midrule
        {Polarisation} & {$0.680(41)$} & {$0.698(27)$} \\   
        {Stromdichte}  & {$0.996(14)$} & {$0.861(22)$} \\
        \bottomrule
        \end{tabular}
\end{table}

\noindent
Die relativen Abweichungen
\begin{equation}
    \Delta x = \left| \frac{x_\text{exp} - x_\text{lit}}{x_\text{lit}}\right|
\end{equation}
der Werte des Polarisationsansatzes vom Literaturwert $W_\text{lit} = \qty{0.66(1)}{\electronvolt}$\cite{Buch} liegen bei:
\begin{align}
    \Delta W_1 \approx \qty{3}{\percent} \\
    \Delta W_2 \approx \qty{6}{\percent}
\end{align}
Bei der ersten Messung liegt der Literaturwert in den Fehlergrenzen, jedoch ist er bei der zweiten Messung knapp außerhalb.
Dies kann unteranderem auch als Rückschluss eines fehlerbehafteten Ablesens am Picoamperemeter gezogen werden (vgl. Abb. \ref{fig:bgr2}).
Dahingegen liegen die relativen Abweichungen der Auswertung mittels des Stromdichtenansatzes bei:
\begin{align}
    \Delta W_1 \approx \qty{51}{\percent} \\
    \Delta W_2 \approx \qty{30}{\percent}
\end{align}
Zudem hat letztere Methode den Literaturwert deutlich verfehlt.
Dieser doch bemerkliche Unterschied der beiden Methoden spiegelt wöhlmöglich die Natur der numerischen Integration gegenüber der
weniger komplexen Methode der Approximation wieder. 

Die Relaxationszeiten unterscheiden sich vor allem durch Nutzung der Aktivierungsenergien des Stromdichtenansatzes (vgl. Tab. \ref{tab:t}).
Bereits Abweichungen von einigen \unit{\milli\electronvolt} führen zu einer Veränderung in der Relaxationszeit von mehreren Größenordnungen.
Die Reproduzierbarkeit der charakteristischen Relaxationszeit unterliegt daher der Genauigkeit der ermittelten Aktivierungsenergien. 
\begin{table}[h]
    \centering
    \caption{Direkter Vergleich der charakteristischen Relaxationszeiten.}
    \label{tab:t}
    \begin{tabular}{cSS}
        \toprule
        {Ansatz} & {Messreihe 1} & {Messreihe 2} \\
        \cmidrule(lr){2-2}  \cmidrule(lr){3-3} 
        & {$\tau \, / \, \si{\second}$} & {$\tau \, / \, \si{\second}$} \\
        \midrule
        {Polarisation} & {$1.06(2.02)e-11$} & {$1.64(1.98)e-11$} \\   
        {Stromdichte}  & {$4.42(2.87)e-18$} & {$1.001(995)e-14$} \\
        \bottomrule
    \end{tabular}
\end{table}
