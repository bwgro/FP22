\section{Theorie}
\label{sec:Theorie}



\subsection{Ionenkristalle}
Ein Ionenkristall setzt sich zusammen aus Kationen (positiv geladen) und Anionen (negativ geladen),
welche in einer Gitterstruktur angeordnet sind. Im Falle von Kaliumbromid ist dies in einer fcc-Struktur.
In einem perfektem Krtistall wären alle Plätze der Struktur durch diese Ionen alternierend besetzt.
Dies würde einen homogenen, elektrisch neutralen Kristall ergeben.
In realen Kristallen hingegen befinden sich immer wieder Störstellen,
die die Struktur des Gitters unterbrechen.
Für diesen Versuch besonders interessant sind die Leerstellen,
also leere Gitterplätze.
Diese können ihre Position im Kristall verändern,
indem Ionen ihren Platz einehmen und einen leeren Platz hinterlassen.

Bei der Dotierung werden dem Ionenkristall Fremdatome im geringen Maße hinzugefügt,
welche den Platz einer Leerstelle einehmen können.
Dabei muss auf mikoskopischer Ebene Ladungsneutralität herrschen.
Sobald das Fremdatom eine Ladungsänderung im Kristall vornimmt,
lösen sich eine entsprechende Anzahl an Ionen aus ihren Gitterplätzen und füllen Leerstellen
oder wandern an die Oberfläche.
Bei einem durch Strontium dotierten Kaliumbromidkristall
ersetzt ein zweifach positiv geladenes Strontium-Ion ein einfach positiv geladenes Kalium-Ion.
Zum mikroskopischen Ladungsausgleich entsteht in der Umgebeung eine Leerstelle 
und das Kation wandert im Festkörper an den Rand.
Leerstelle und Dotierung bilden hier einen Dipol.



\subsection{Dipole und Dipolrelaxion in Ionenkristallen}
Ein Dipol ist die Anordnung zweier entgegengesetzter Ladungen.
Das zugehörige Dipolmoment $\vec{p}$ berechnet sich durch
\begin{equation*}
    \vec{p} = \sum_i q_i \cdot \vec{r_i}
\end{equation*}
mit den Ladungen $q_i$ an den Orten $\vec{r_i}$ 
und zeigt immer in Richtung der positiven Ladung.
Innerhalb eines elektrischen Feldes richten sich diese Dipole entlang der Feldlinien aus,
um einen Zustand minimaler Energie einnehmen zu können.
Daduch bildet sich eine Vorzugsrichtung für das Dipolmoment.
Ohne elektrisches Feld sind die Dipole so verteilt, 
dass sich ihr Gesammtdipolmoment zu Null addiert.

Die Zeit welche die Dipole nach der Polarisation benötigen,
um eine Richtungsänderung zu vollführen nennt sich Relaxionszeit:
\begin{equation}
    \tau(T) = \tau_0 \cdot \exp\left(\frac{W}{k_B T}\right) , \tau_0 =\tau(\infty).
\end{equation}
Sie ist abhängig von der Aktivierungsenergie $W$,
welche benötigt wird,
um die Coulomb-Barriere des Ionengitters zu überwinden,
so wie der Temperatur $T$.
Zu erkennen ist,
dass die Relaxionszeit bei niedrigen Temperaturen niedrig und bei hohen hoch ist.
Daher wird der Ionenkristall bei etwa \qty{320}{\kelvin} in ein E-Feld gegeben,
damit die Einschaltdauer möglichst groß gegenüber der Relaxationszeit ist und eine Polarisation im Kristall möglich ist.
Durch das anschließende Abkühlen wird diese Polarisation auch bei abgeschaltetem E-Feld quasi festgehalten. 



\subsection{Der Depolarisationsstrom}
Bei der stattfindenden Depolarisation wird ein Strom induziert,
welcher gemessen werden kann. 
Aus diesem Depolarisationsstrom lassen sich die Größen Aktivierungsenergie und Relaxionszeit bestimmen.
Ermittelt werden kann der Depolarisationsstrom durch zwei Ansätze.

\subsubsection{Polarisationsansatz}
Der Depolarisationsstrom beträgt dabei im allgemeinen
\begin{equation}
    i(T) = -\frac{\mathrm{d}P(t)}{\mathrm{d}t}\,.
    \label{eqn:polstrom2}
\end{equation}
Die Polarisationsrate hängt von der übrigen Polarisation zur Zeit $t$
und von der Relaxationsrate $\tau(T)$ ab.
\begin{equation}
    \frac{\text{d}{P(t)}}{\text{d}{t}} = -\frac{P(t)}{\tau(T)}
    \label{eqn:polrate}
\end{equation}
Aus den beiden Gleichungen \eqref{eqn:polrate} und \eqref{eqn:polstrom2} ergibt sich daher
\begin{equation}
    i(T) = \frac{P(t)}{\tau(T)}\,.
    \label{eqn:i_t_polstart}
\end{equation}
Gleichung \eqref{eqn:polrate} integriert führt zu einen Zusammenhang für $P(t)$
\begin{equation}
    P(t) = P_{0} \exp\!\left(-\frac{t}{\tau(T)}\right)\,,
\end{equation}
welcher sich mit Gleichung \eqref{eqn:i_t_polstart} zu
\begin{equation}
    i(T) = \frac{P_0}{\tau} \exp\!\left(-\frac{t}{\tau(T)}\right) 
\end{equation}
ergibt.
Die Zeit $t$ wird nun als Integral über den Startpunkt bis zum Beginn des Depolarisationsstroms beschrieben
\begin{equation}
    i(T) = \frac{P_{0}}{\tau} \exp\!\left(-\int_0^t \frac{\text{d}{t}}{\tau(T)}\right)\,.
\end{equation}
Bei einer konstanten Heizrate $b$ nimmt der der Depolarisationsstrom die Form
\begin{equation}
    i(T) = \frac{P_0}{\tau} \exp\!\left(-\int_{T_0}^T
      \exp\!\left(- \frac{W}{k_B T}\right) \text{d}{T}\right)
    \label{eq:final}
\end{equation}
an.

Zur Berechnung der Aktivierungsenergie $W$ und der Relaxationszeit wird angenommen,
dass $W$ groß im Vergleich zur thermischen Energie $k_B T$
und die Temperaturdiferenz $T - T_0$ gering ist.
Mit der Annahme wird aus \refeq{eq:final}
\begin{equation}
    i(T) = \frac{P_0}{\tau} \exp\!\left(-\int_{T_0}^T
      \exp\!\left(- \frac{W}{k_B T}\right) \text{d}{T}\right) \approx 0
\end{equation}
und der Ausdruck für den Strom vereinfacht sich zu 
\begin{equation}
    I(T) \approx \frac{p^2E}{3k_B T_0}\cdot\frac{N_0}{\tau_0}\cdot\exp{\frac{-W}{k_B T}}.
\end{equation}
Mit dem Bilden des Logarithmus ergibt sich
\begin{equation}
    \log(I(T)) = \text{const} -\frac{W}{k_B T}
    \label{eq:W1}
\end{equation}
was in der Auswertung für eine lineare Ausgleichsrechnung zwischen $\log(I)$
und $T^{-1}$ zur Bestimmung von W verwendet wird.
Für die Relaxationszeit gilt am Maximum 
\begin{equation}
    \tau_{\text{max}} (T_{\text{max}}) = \frac{k_\text{B} T^2_{\text{max}}}{b W}\, ,
\end{equation}
woraus die charakteristische Relaxationszeit $\tau_0$ 
\begin{equation}
    \tau_0 = \tau_{\text{max}} (T_{\text{max}}) \exp \left(- \frac{W}{k_{\text{B}} \cdot T_\text{max}}\right) = \frac{k_\text{B} T^2_{\text{max}}}{b W} \cdot\exp \left(- \frac{W}{k_{\text{B}} \cdot T_\text{max}}\right)
    \label{eq:relax}
\end{equation}
berechnet werden kann.


\subsubsection{Stromdichtenansatz}
Der zweite Ansatz zur Bestimmung der Aktivierungsenergie $W$ geht über die Annahme,
dass die Änderung der Polarisation $P$ mit der Zeit zum Betrag der Stromdichte $j(T)$
entspricht.
Die Änderung der Polarisation $P$ ist definiert als
\begin{equation*}
    \frac{\text{d}P}{\text{d}t} = -\frac{P(t)}{\tau(T)}\, .
\end{equation*}
Nach umstellen und erweitern mit $\frac{\text{d}T}{\text{d}T}$ folgt 
\begin{equation*}
    \tau(T) = P(T) \cdot \frac{\text{d}T}{\frac{\text{d}P}{\text{d}t}\text{d}T } =  \frac{P(t)}{b} \frac{\text{d}T}{\text{d}P}\,.
\end{equation*}
Durch erneute Erweiterung mit $\frac{\text{d}t}{\text{d}t}$ ergibt sich:
\begin{equation*}
    \tau(T) = \frac{P(t)}{b} \frac{\frac{\text{d}T}{\text{d}t}}{\frac{\text{d}P}{\text{d}t}}
\end{equation*}
Nun wird durch die vorhergehende Annahme und $P =\int\text{d}P$ der Term für die Relaxationszeit umgeformt zu: 
\begin{equation*}
    \tau(T) = \frac{\int \frac{\text{d}P}{\text{d}t} \text{d}T}{I(T) \cdot b} = \frac{\int_T^\infty I(T') \, \text{d}T'}{I(T) \cdot b}
\end{equation*}
Einsetzen und die Gleichung zur Aktivierungsenergie $W$ umstellen ergibt:
\begin{equation}
    W = k_\text{B} T \cdot \ln\left( \frac{\int_T^\infty I(T') \, \text{d}T'}{b \tau_0 \cdot I(T)}\right)
    \label{eq:Int}
\end{equation}
Somit lässt sich nun die Aktivierungsenergie $W$ berechnen. In der Praxis wird die obere Integrationsgrenze von $\infty$ in $T^*$ geändert, mit
$I\left(T^*\right)\approx 0$. Demnach soll $T^*$ groß genug sein, um eine Gleichverteilung der Dipole hervorzurufen. 