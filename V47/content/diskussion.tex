\section{Diskussion}
\label{sec:Diskussion}
Für die Debye-Temperatur wurden für Kupfer die Werte
\begin{align*}
    \Theta_{D, \text{Messung}} &= 443,08K \\
    \Theta_{D, \text{Theorie}} &= 331,99K \\
\end{align*}
bestimmt.
Mit dem Literaturwert \cite{Lit}
\begin{equation*}
    \Theta_D = 345K,
\end{equation*}
ergeben sich die Abweichungen
\begin{align*}
    \Delta\Theta_{D, \text{Messung}} &= 22,14\%, \\
    \Delta\Theta_{D, \text{Theorie}} &= 4,05 \%.
\end{align*}
Dies sind vertretbare Abweichungen für die Methode der Messung.
Mögliche Ursachen für die Abweichung sind,
dass eine Angleichung der Temperaturen von Probe und Probengehäuse notwendig ist, 
um Verluste durch Wärmestrahlung zu unterdrücken. 
Da das Heizsystem des Gehäuses nicht an die Probenheizung angepasst arbeitet, 
muss es von Hand eingestellt und gegebenenfalls nachjustiert werden. 
Weiterhin reagiert das System träge auf Leistungsanpassungen. 
Es konnte daher die obige Forderung nicht mit Sicherheit gewährleistet werden und somit ist von störenden Effekten, 
die das Ergebnis beeinflussen, 
auszugehen. 
Außerdem kann der Austausch von Wärme durch Konvektion nicht ganz ausgeschlossen werden, 
da im Rezipienten kein totales Vakuum herrscht.